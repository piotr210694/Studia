\documentclass[a4paper, 12pt]{article}
\pagestyle{plain}
\inputenc[cp1250]{inputenc}
\usepackage{polski}
 \renewcommand{\labelitemi}{$\blacksquare$}
 \renewcommand\labelitemii{$\square$}

\begin{document}

	\tableofcontents

	\section{Pierwsza sekcja}
		Hello World
		
		\subsection{Podsekcja pierwszej sekcji}
			Dziala?

	\section{Druga sekcja}
		Ala ma kota. Kot ma Al�.

	\begin{enumerate}
		\item
			Pierwszy item
		\item
			Drugi item
	\end{enumerate}
	
	\begin{enumerate}
	\end{enumerate}
	
	\begin{description}
	\item[pogrubione]
	definicja
	\end{description}
	
	\paragraph{nowy paragraf}
	A tu tekst
	
	\section{jakas nowa sekcja}
	\section{sekcja pomiedzy}
	
	Ala ma {\bf kota}.
	Ala ma {\bfseries kota}.
	Ala ma \textbf {kota}.
	
	\begin{itemize}
	\item  Default item label for entry one
	\item[$\square$]  Custom item label for entry three
	\end{itemize}
	
	\begin{itemize}
 \renewcommand{\labelitemi}{\scriptsize$\blacksquare$} 
 \item  An extensively analysis 
\end{itemize}
	
	%lab3
	%\rule{0pt}{10pt}
	%\vspace{10cm}
	
	\nonindent
	{\tiny %\rule{1em}{10pt}
	Ala ma \textbf{kota i psa}.}\\
	{\scriptsize Ala ma \textbf {kota i psa}}\\
	{\footnotesize Ala ma \textbf {kota i psa}}\\
	{\small Ala ma kota}
	{Ala ma \textbf{kota \footnote{Kot nie lubi ali.} i psa . }\\
	{\large Ala ma kota}\\
	{\Large Ala ma \textbf{kota}\footnote{A to inny footnote}}\\
	{\huge Ala ma}\\
	
	
\end{document}